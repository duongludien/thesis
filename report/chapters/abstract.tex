\documentclass[../thesis.tex]{subfiles}

\begin{document}


\begin{centering}
	\section*{Tóm tắt nội dung}
\end{centering}

Trong đề tài này, chúng tôi đề xuất sử dụng YOLO, một thuật toán tiên tiến trong việc giải quyết bài toán nhận dạng đối tượng trong ảnh vào bài toán nhận dạng biển báo giao thông tại Việt Nam. Chúng tôi đã tiến hành thực nghiệm và đạt được điểm F1 là 92\% trên tập dữ liệu khoảng 5000 ảnh chứa 22 loại biển báo giao thông do chúng tôi tự thu thập. 

Ngoài ra, chúng tôi cũng cài đặt lại thuật toán YOLO trên TensorFlow, một framework cho học tập sâu được phát triển bởi Google giúp cho thời gian nhận dạng nhanh gấp 3 lần so với phiên bản gốc của thuật toán được cài đặt bằng ngôn ngữ C và sẵn sàng để triển khai trên các ứng dụng thực tế.

\begin{centering}
	\section*{Abstract}
\end{centering}

In this work, we propose using YOLO, a state-of-the-art object detection algorithm for detecting and classifying traffic-signs in Vietnam. We have experimented and achieved the F1 score at 92\% for our dataset which contains about 5000 images of 22 types of traffic-signs.

Moreover, we also implemented YOLO in TensorFlow, a deep learning framework which is developed by Google. It helps our model run 3 times faster than the original version which was implemented in C and ready for production.

\end{document}
