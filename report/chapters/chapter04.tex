\documentclass[../thesis.tex]{subfiles}

\begin{document}

\section{Kết luận}

Mô hình đạt được mục tiêu đề ra ban đầu là trung hòa giữa độ chính xác và thời gian nhận dạng. Tuy nhiên, mô hình rút gọn vẫn mắc phải những điểm hạn chế như không thể nhận dạng được các đối tượng quá nhỏ. Hầu hết các đối tượng thuộc lớp \textit{traffic\_light} đều không nhận dạng được. Không chỉ riêng về mô hình, tập dữ liệu cũng ảnh hưởng một phần không nhỏ đến chất lượng của mô hình, kích thước tập dữ liệu vẫn chưa thực sự lớn, một số lớp vẫn còn thiếu dữ liệu, một số khung ảnh được chọn chứa các đối tượng quá nhỏ, việc chọn camera góc rộng là nguyên nhân chính dẫn đến vấn đề này.

\section{Hướng phát triển}

Trong thời gian tới, chúng tôi sẽ tiếp tục thu thập thêm dữ liệu biển báo cho các lớp bị thiếu dữ liệu và các lớp chưa có trong tập dữ liệu, bổ sung dữ liệu cho các lớp đã có. Hướng tới một mô hình nhận dạng đối tượng hiệu quả cho xe tự hành. Ngoài ra, chúng tôi cũng sẽ tiếp tục cài đặt các đoạn script huấn luyện mô hình bằng TensorFlow để tăng hiệu năng lúc huấn luyện. 

\end{document}